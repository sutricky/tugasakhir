\begin{center}
       \Large
       \textbf{ABSTRAK} \\
        \vspace{1.5cm}
        \large
        \textbf{Pemodelan Termal Semi-empiris Satelit LAPAN-A3 Menggunakan Metode Machine Learning}\\
        
        \vspace{1.5cm}
        
        oleh\\
        Ricky Sutardi\\
        NIM : 13617051\\
        (Program Studi Sarjana Teknik Dirgantara)\\
        \vspace{1.5cm}
\end{center}

Dari pembacaan sensor suhu satelit, ditemukan bahwa LAPAN-A3, sebuah
mikrosatelit dari Indonesia, membutuhkan maneuver khusus untuk menjaga suhu
\textit{payload} utamnya, sebuah pencitra multispektral, tetap di atas 0
\degree C. Dapat disimpulkan bahwa sistem kendali termal pasif satelit tersebut
tidak cukup untuk memenuhi kebutuhan misinya. Agar permasalahan ini tidak
terjadi kembali di desain termal satelit masa depan, dibutuhkan sebuah model
termal sederhana baru yang dapat memodelkan karakteristik termal satelit
LAPAN-A3 dengan akurat.

Analisis termal satelit konvensional membutuhkan perhitungan banyak variabel
secara akurat sehingga memiliki kompleksitas yang tinggi pula. Sementara itu,
pendekatan berbasis data dalam menyelesaikan permasalahan termal
satelit semakin umum digunakan untuk mengurangi kompleksitas pemodelan termal
satelit. Karena itu, model termal sederhana yang dapat menghitung
parameter-parameter termal satelit yang dibutuhkan untuk memprediksi suhu
satelit dari data operasional satelit diharapkan dapat menghasilkan proses
pemodelan termal yang lebih cepat dan mudah.

Dalam karya tulis ini, model termal semi-empiris LAPAN-A3 dengan 7
\textit{node} dikembangkan dengan metode \textit{machine learning}. Model
tersebut dilatih dengan data satelit dari periode observasi 19 dan 20 Mei 2018.
Metode regresi linear menggunakan \textit{machine learning} dipilih untuk
mengurangi jumlah variabel yang harus dihitung untuk memodelkan satelit.
Evaluasi awal menunjukkan model termal secara umum dapat memprediksi perubahan
suhu \textit{node} satelit serta memiliki potensi kegunaan nyata dalam
pengembangan satelit. Meski masih dapat dikembangkan lebih lanjut, model
tersebut dapat digunakan sebagai dasar model termal satelit masa depan. 

\vspace{1.0cm}
\noindent 
Kata kunci : model termal, satelit, machine learning, regresi linear, LAPAN-A3

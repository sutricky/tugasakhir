\begin{center}
       \Large
       \textbf{ABSTRAK} \\
        \vspace{1.5cm}
        \large
        \textbf{Pemodelan Termal Semi-empiris Satelit LAPAN-A3 Menggunakan Metode Machine Learning}\\
        
        \vspace{1.5cm}
        
        oleh\\
        Ricky Sutardi\\
        NIM : 13617051\\
        (Program Studi Sarjana Teknik Dirgantara)\\
        \vspace{1.5cm}
\end{center}

Dari pembacaan sensor suhu satelit, ditemukan bahwa LAPAN-A3, sebuah
mikrosatelit dari Indonesia, membutuhkan maneuver khusus untuk menjaga suhu
\textit{payload} utamanya, sebuah pencitra multispektral, tetap di atas 0
\degree C. Dapat disimpulkan bahwa sistem kendali termal pasif satelit tersebut
tidak cukup untuk memenuhi kebutuhan misinya. Agar permasalahan ini tidak
terjadi kembali di desain termal satelit masa depan, dibutuhkan sebuah model
termal sederhana baru yang dapat memodelkan karakteristik termal satelit
LAPAN-A3 dengan akurat.

Analisis termal satelit konvensional membutuhkan perhitungan banyak variabel
secara akurat sehingga memiliki kompleksitas yang tinggi pula. Sementara itu,
pendekatan berbasis data dalam menyelesaikan permasalahan termal satelit
semakin umum digunakan untuk mengurangi kompleksitas dan jumlah perhitungan
variabel pemodelan termal satelit. Karena itu, pendekatan semi-empiris yang
menggunakan data empiris telemetri dan observasi satelit untuk menyederhanakan
perhitungan teoretik persamaan termal satelit diharapkan dapat menghasilkan
proses pemodelan termal yang lebih cepat dan mudah.

Dalam karya tulis ini, model termal semi-empiris LAPAN-A3 dengan 7
\textit{node} dikembangkan dengan metode \textit{machine learning}
lewat pendekatan regresi linear. Model tersebut dilatih dan diuji
dengan data operasional satelit LAPAN-A3 dari periode observasi 19 dan
20 Mei 2018 untuk menghitung variabel-variabel yang belum diketahui
pada persamaan termal satelit LAPAN-A3 sehingga dapat memprediksi
perubahan suhu \textit{node-node} satelit LAPAN-A3. Evaluasi awal
menunjukkan model termal secara umum dapat memprediksi perubahan suhu
\textit{node} satelit serta memiliki potensi kegunaan nyata dalam
pengembangan satelit. Meski masih dapat dikembangkan lebih lanjut,
model tersebut dapat digunakan sebagai dasar model termal satelit masa
depan. 

\vspace{1.0cm}
\noindent 
Kata kunci : model termal, satelit, machine learning, regresi linear, LAPAN-A3

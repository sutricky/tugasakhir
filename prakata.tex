Puji syukur penulis panjatkan kepada Tuhan Yang Maha Esa karena hanya berkat
rahmat-Nya karya tulis ini dapat diselesaikan. Karya tulis ini dibuat sebagai
syarat menyelesaikan studi pada program sarjana Teknik Dirgantara ITB. Proses
pembuatan karya tulis ini tidak lepas dari bantuan serta dukungan banyak pihak
kepada penulis. Karena itu, penulis berterima kasih kepada :

\begin{enumerate}
\item Tuhan Yang Maha Esa atas berkat kesehatan dan hikmat yang telah diberikan kepada penulis sehingga mampu menyelesaikan karya tulis ini dengan selamat.
\item Bapak Dr. Eng Ridanto Eko Poetro, Dr. Robertus Heru Triharjanto, dan Luqman Fathurrohim selaku dosen-dosen pembimbing atas semua kesabaran dalam membimbing penulis serta menjawab pertanyaan-pertanyaan yang muncul selama pengerjaan tugas akhir.
\item Keluarga yang tetap percaya penulis dapat menyelesaikan tugas akhir ini serta memberikan dukungan dan bantuan kepada penulis.
\item Teman-teman Ganesha Ipsun 17 yang menjadi pemicu semangat penulis untuk menutup kelulusan angkatan. 
\item Icarus AE17 dan seluruh massa KMPN yang menjadi rumah bagi penulis untuk tetap dapat pulang setelah lelah terbang.
\end{enumerate}

Tentu tugas akhir ini masih memiliki banyak kekurangan akibat keterbatasan
pengetahuan yang dimiliki oleh penulis. Masih banyak hal juga yang masih masih
bisa ditingkatkan dalam penelitian-penelitian selanjutnya. Oleh karena itu,
penulis berharap masukan berupa saran dan kritik dari pembaca untuk terus
memperbaiki karya tulis ini.

Terakhir, semoga karya tulis ini dapat menjadi pemrakarsa dan pemicu
pengembangan lebih lanjut dalam bidang satelit di Indonesia. Penulis juga
berharap para pembaca mendapatkan tambahan wawasan setelah membaca karya tulis
ini. Demi dunia dirgantara yang lebih baik lagi, merdeka!

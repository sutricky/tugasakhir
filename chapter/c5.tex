\chapter{Kesimpulan dan Saran}

\section{Kesimpulan}

Dari penelitian yang dilakukan pada karya tulis ini, didapatkan langkah-langkah
pemodelan termal semi-empiris satelit menggunakan metode \textit{machine learning}.
Algoritma pemodelan termal tersebut diimplementasikan pada model satelit 7 \textit{node}
LAPAN-A3 melalui bahasa pemrograman Python serta dilatih dengan data telemetri
operasi LAPAN-A3 aktual dari 19 sampai dengan 20 Mei 2018. Hasilnya adalah model
termal yang dapat memprediksi suhu satelit LAPAN-A3 selama periode
observasi.

Secara umum, model termal yang dihasilkan dapat memprediksi tren perubahan suhu satelit LAPAN-A3 dengan akurasi yang cukup baik. Hal ini dapat
dilihat dari nilai koefisien determinasi model termal yang lebih besar dari 0.9
untuk 6 dari 7 \textit{node} pada periode observasi 19 Mei 2018 dan 5 dari 7 \textit{node} untuk
20 Mei 2018. Selain itu, 6 dari 7 \textit{node} model termal menghasilkan skor
\textit{root mean square error} kurang dari 1 \degree C untuk kedua periode observasi. 

Meski memiliki beberapa catatan terkait perbaikan performa, model termal yang
dikembangkan di karya tulis ini sudah memenuhi tujuan pembuatan karya tulis
ini. Hasil yang dicapai iterasi model termal saat ini akan menjadi
\textit{baseline} untuk performa model termal satelit di masa depan. Dengan
demikian, performa model termal satelit yang dikembangkan dapat terus bertambah
semakin akurat.

\section{Saran}

Berikut adalah beberapa saran untuk melanjutkan penelitian karya tulis ini :

\begin{enumerate}
\item Menambah jumlah \textit{node} yang digunakan dalam pemodelan sehingga karakteristik termal satelit dapat terwakili lebih menyeluruh
\item Mengubah asumsi yang digunakan pada \textit{node} satelit untuk memodelkan karakteristik \textit{node} satelit lebih akurat
\item Melakukan analisis lebih lanjut untuk menghitung kontribusi tiap variabel dalam persamaan termal satelit
\item Memperpanjang durasi periode observasi sehingga masukan data yang dapat digunakan juga bertambah
\item Menggunakan data dari periode observasi saat satelit melakukan maneuver agar model termal dapat mengakomodasi efek perubahan sikap satelit lebih baik
\end{enumerate}

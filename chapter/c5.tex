\chapter{Kesimpulan dan Saran}

\section{Kesimpulan}

Dari penelitian yang dilakukan pada karya tulis ini, didapatkan langkah-langkah
pemodelan termal semi-empiris satelit menggunakan metode Machine Learning.
Algoritma pemodelan termal tersebut diimplementasikan pada model satelit 7 node
LAPAN-A3 melalui bahasa pemrograman Python serta dilatih dengan data telemetri
operasi LAPAN-A3 aktual dari 19 sampai dengan 20 Mei 2018. Hasilnya adalah model
termal yang dapat memprediksi suhu node-node satelit LAPAN-A3 selama periode
observasi.

Secara umum, model termal yang dihasilkan dapat memprediksi tren perubahan suhu
node-node satelit LAPAN-A3 dengan akurasi yang cukup baik. Hal ini dapat
dilihat dari nilai koefisien determinasi model termal yang lebih besar dari 0.9
untuk 6 dari 7 node pada periode observasi 19 Mei 2018 dan 5 dari 7 node untuk
20 Mei 2018. Selain itu, 6 dari 7 node model termal menghasilkan skor
\textit{root mean square error} kurang dari 1 \degree C untuk kedua periode observasi.

\section{Saran}

Berikut adalah beberapa saran untuk melanjutkan penelitian karya tulis ini :

\begin{enumerate}
\item Menambah jumlah node yang dimodelkan sehingga karakteristik termal satelit dapat termodelkan lebih menyeluruh
\item Mengubah asumsi yang digunakan pada node satelit untuk memodelkan perilaku node satelit lebih akurat
\item Melakukan analisis lebih lanjut untuk menghitung kontribusi tiap faktor termal satelit
\item Memperpanjang durasi periode observasi sehingga masukan data yang dapat digunakan juga bertambah
\item Menggunakan data dari periode observasi saat satelit melakukan maneuver agar model termal dapat mengakomodasi efek perubahan sikap satelit lebih baik
\end{enumerate}

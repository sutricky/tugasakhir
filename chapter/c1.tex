\chapter{Pendahuluan}
\pagenumbering{arabic}

\section{Latar Belakang}

Pengendalian termal satelit bertujuan untuk menjaga semua komponen satelit
tetap pada rentang suhu operasional pada selama masa misi satelit. Desain
sistem kendali termal yang buruk dapat mengakibatkan kerusakan permanen pada
komponen satelit akibat panas berlebih atau kondisi non-operasional yang fatal
dalam misi akibat suhu komponen yang terlalu rendah. Semakin dekat model termal
satelit dengan karakteristik termal satelit yang sebenarnya, semakin kecil pula
kemungkinan suhu satelit mencapai nilai di luar rentang yang sudah ditentukan.
Karena itu, desain sistem kendali termal satelit yang baik membutuhkan
pemodelan termal satelit yang baik pula. Hal ini membuat proses desain sistem
kendali termal satelit menjadi salah satu tahap yang paling penting dalam
periode pengembangan satelit.

Permasalahan yang mungkin dihadapi pada proses desain sistem termal satelit
antara lain muncul karena karakteristik termal satelit ditentukan oleh banyak
parameter yang saling bergantung sama lain. Untuk memodelkan karakteristik
termal satelit secara akurat, tentunya dibutuhkan perhitungan parameter satelit
yang akurat juga. Selain itu, desain sistem kendali termal satelit harus turut
memperhitungkan batasan satelit lain seperti ruang, daya, dan berat.

Pada kasus satelit mikro dan nano, ukuran satelit yang kecil dapat berakibat
pada kapasitas termal yang kecil pula sehingga satelit mudah mengalami
perubahan suhu. Satelit mikro dan nano juga biasanya hanya bergantung pada
sistem kendali termal pasif akibat batasan ketersediaan daya. Akibatnya,
seringkali diperlukan langkah tambahan pada saat operasi untuk memastikan
satelit dapat tetap dijaga pada rentang suhu operasional. 

Permasalahan di atas terjadi pada LAPAN-A3, satelit yang memiliki misi utama
observasi Bumi serta diluncurkan pada tahun 2016. LAPAN-A3 merupakan satelit
yang dikembangkan Lembaga Penerbangan dan Antariksa Nasional (LAPAN) dengan
kerja sama dengan Institut Pertanian Bogor (IPB) setelah sebelumnya LAPAN telah
berhasil meluncurkan satelit LAPAN-A1 dan LAPAN-A2. Ketiga satelit tesebut
termasuk dalam kategori mikrosatelit serta menggunakan sistem kendali termal
pasif dengan cara distribusi panas lewat struktur dan penggunaan cat khusus
untuk memantulkan radiasi. Berbeda dengan LAPAN-A1 dan LAPAN-A2,
LAPAN-A3 membutuhkan maneuver khusus untuk menjaga suhu \textit{payload}
utamanya tetap pada rentang suhu operasional.

LAPAN-A3 memiliki \textit{payload} utama berupa pencitra multispektral yang
harus dijaga pada rentang suhu operasional 0 \degree C sampai dengan 70 \degree
C. Dari hasil observasi data telemetri, ditemukan bahwa suhu pencitra dapat
turun hingga nilai negatif dan berada pada rentang -2 \degree C dan -3 \degree
C pada periode bulan Mei sampai dengan Juni seperti yang dapat dilihat pada
Gambar \ref{fig:multispectralimager} \cite{ribah2019}. Kondisi ini berdampak
buruk pada misi satelit LAPAN-A3 karena pencitra tidak dapat dioperasikan pada
suhu di bawah 0 \degree C. Akibatnya, frekuensi misi pengambilan citra Bumi
harus dikurangi sedangkan hal tersebut merupakan misi utama dari satelit LAPAN-A3.

\begin{figure}[H]
\setlength\belowcaptionskip{-0.7\baselineskip}
\begin{center}
\includegraphics[width=0.9\textwidth]{fig/multispectralimager.png}
	\caption[Grafik suhu \textit{multispectral imager} LAPAN-A3 sepanjang tahun 2017]{Grafik suhu \textit{multispectral imager} LAPAN-A3 sepanjang tahun 2017~\cite{ribah2019}}
\label{fig:multispectralimager}
\end{center}
\end{figure}

Analisis lebih lanjut dari simulasi orbit satelit menemukan hubungan fenomena
penurunan suhu pencitra dengan variasi sudut antara bidang orbit satelit dan
vektor Matahari (sudut $\beta$ orbit) akibat perubahan musim seperti yang dapat dilihat pada Gambar \ref{fig:betaorbit}.

\begin{figure}[H]
\setlength\belowcaptionskip{-0.7\baselineskip}
\begin{center}
\includegraphics[width=0.7\textwidth]{fig/betaorbit.png}
	\caption[Ilustrasi sudut $\beta$ orbit satelit]{Ilustrasi sudut $\beta$ orbit satelit~\cite{2012a}}
\label{fig:betaorbit}
\end{center}
\end{figure}

Untuk menaikkan suhu pencitra dan melawan penurunan sudut $\beta$ tersebut,
operator LAPAN-A3 memberikan perintah maneuver \textit{stop and release} serta
\textit{stop and release with roll} seperti yang ditunjukkan Gambar \ref{fig:maneuver1} dan \ref{fig:maneuver2} sehingga sisi satelit yang memuat pencitra
menerima paparan sinar Matahari lebih banyak. Periode pemberian perintah
maneuver tidak dilakukan secara acak, tapi diatur sedemikian rupa agar satelit
dapat kembali pada sikap \textit{nadir pointing} di akhir maneuver.

\begin{figure}[H]
\setlength\belowcaptionskip{-0.7\baselineskip}
\begin{center}
\includegraphics[width=0.9\textwidth]{fig/maneuver1.png}
	\caption[Ilustrasi maneuver \textit{stop and release}]{Ilustrasi maneuver \textit{stop and release}~\cite{ribah2019}}
\label{fig:maneuver1}
\end{center}
\end{figure}

\begin{figure}[H]
\setlength\belowcaptionskip{-0.7\baselineskip}
\begin{center}
\includegraphics[width=0.9\textwidth]{fig/maneuver2.png}
	\caption[Ilustrasi maneuver \textit{stop and release with roll}]{Ilustrasi maneuver \textit{stop and release with roll}~\cite{ribah2019}}
\label{fig:maneuver2}
\end{center}
\end{figure}

Jelas dapat dilihat bahwa perlindungan termal pasif pada struktur LAPAN-A3
berupa alumunium yang telah diberi anode hitam sendiri saja tidak dapat
memenuhi persyaratan termal misi LAPAN-A3. Tanpa maneuver spesial, satelit
tidak dapat menjalankan misi dengan optimal selama bulan Mei sampai dengan
Juni. Selain itu, penggunaan maneuver khusus sebagai solusi tetap memiliki
konsekuensi negatif terhadap operasi harian LAPAN-A3.

Pertama, pencitra harus dimatikan selama durasi maneuver yang berkisar antara
90 sampai dengan 280 menit. Hal ini tetap berakibat pada berkurangnya frekuensi
misi pengambilan citra meski tidak sebanyak pengurangan yang terjadi jika
pencitra harus dimatikan selama bulan Mei sampai dengan Juni. Kemudian,
maneuver khusus membuat satelit terkunci dalam sikap \textit{off-nadir
pointing} selama durasi maneuver. Akibatnya, misi yang membutuhkan sikap
\textit{nadir pointing} tidak bisa dilakukan sampai maneuver berakhir dan harus
dikompensasi dengan misi lain yang tidak membutuhkan \textit{nadir pointing}
seperti pengukuran medan magnet Bumi.

Agar permasalahan yang terjadi pada satelit LAPAN-A3 saat ini tidak terulangi
pada desain satelit-satelit LAPAN di masa depan, dibutuhkan sebuah model termal
satelit yang dapat digunakan sebagai referensi dalam pengembangan
satelit-satelit LAPAN selanjutnya. Pemodelan termal satelit konvensional
dilakukan dengan menyelesaikan persamaan termal satelit yang sudah dimodelkan
menjadi beberapa titik analisis atau \textit{node} lewat perhitungan numerik
atau simulasi perangkat lunak komersial. Untuk satelit dalam kelas mikro, Das
et al. telah mengembangkan prosedur desain termal satelit sederhana menggunakan
pendekatan pertama lewat penyelesaian persamaan termal satelit \textit{node}
banyak yang disederhanakan \cite{das}. Kemudian, pendekatan kedua sudah
dilakukan oleh Boudjemai et al. yang menggunakan metode elemen hingga untuk
menganalisis karakteristik termal baterai satelit \cite{boudjemai2015}.

Karena LAPAN belum memiliki perangkat lunak yang dibutuhkan
\cite{budiantoro2019}, metode yang mungkin dipakai hanya metode perhitungan
numerik. Pendekatan ini membutuhkan perhitungan banyak parameter satelit secara
akurat bahkan untuk model satelit 1 \textit{node} sekalipun. Akibatnya,
penggunaan metode perhitungan numerik biasanya terbatas untuk model satelit 1,
2, atau 3 \textit{node} saja. Hal ini menjadi permasalahan yang harus
diselesaikan karena model termal satelit yang akan dikembangkan diharapkan
dapat memprediksi sisi-sisi satelit LAPAN-A3 sehingga harus menggunakan model
satelit dengan jumlah \textit{node} setidaknya sesuai jumlah sisi satelit
LAPAN-A3.

Untuk model satelit \textit{node} banyak, pendekatan yang umum digunakan adalah
metode kedua berupa simulasi lewat perangkat lunak komersial. Sebagai contoh,
Totani et al. menjelaskan prosedur desain termal satelit kelas mikro dan nano
menggunakan metode perhitungan numerik untuk pendekatan satelit 1 dan 2
\textit{node} saja \cite{totani2014}. Pada jurnal yang sama, prosedur
desain termal satelit dengan pendekatan model \textit{node} banyak dilakukan
dengan bantuan perangkat lunak SINDA/FLUINT. Selain itu, sampai karya tulis ini
dibuat, ketiga karya tulis yang telah disebutkan belum memberikan validasi
metode yang diajukan dengan data telemetri aktual satelit. Hal ini cukup
disayangkan karena LAPAN-A3 sudah mengorbit Bumi sejak 2016 sehingga terdapat
banyak sekali data historis operasional yang seharusnya dapat digunakan dalam
pemodelan termal satelit LAPAN-A3.

Berdasarkan alasan yang telah dijelaskan di atas, penelitian pada karya tulis
ini bertujuan untuk menjabarkan langkah-langkah pembuatan model termal satelit
sederhana yang dapat mengurangi jumlah perhitungan serta dapat divalidasi
dengan data telemetri satelit. Kemudian, prosedur pembuatan model termal
tersebut akan diimplementasikan pada satelit LAPAN-A3. Diharapkan bahwa model
termal yang dihasilkan pada karya tulis ini dapat memprediksi suhu satelit
secara akurat sehingga satelit LAPAN di masa depan dapat memiliki prediksi suhu
komponen yang lebih akurat juga. Model termal yang lebih akurat juga akan
memungkinkan lebih banyak misi pencitraan multispektral seperti dalam LAPAN-A4
atau \textit{payload} yang mengeluarkan panas seperti \textit{Synthetic
Apperture Radar} (SAR) pada LAPAN-A5/Chibasat.

Pada karya tulis ini, metode regresi linear berbasis \textit{machine learning}
dalam bahasa pemrograman Python akan digunakan untuk membuat model termal
satelit LAPAN-A3 secara semi-empiris. Pemodelan termal satelit semi-empiris
dilakukan dengan menggunakan asumsi, pendekatan, dan generalisasi untuk
menyederhanakan perhitungan teoretis berdasarkan hasil observasi pada satelit.
Lebih lanjut, data telemetri satelit juga digunakan dalam perhitungan
variabel-variabel dalam persamaan termal satelit sehingga jumlah variabel yang
harus dihitung dapat berkurang secara signifikan. Selain itu, lewat metode
\textit{machine learning}, proses perhitungan variabel-variabel tersebut dapat
diotomasi secara cepat dan mudah.

Penggunaan metode \textit{machine learning} bukanlah fenomena baru dalam
analisis termal satelit. Sebagai contoh, metode \textit{machine learning} telah
digunakan untuk membuat simulasi termal satelit \textit{real-time}
\cite{junior2017}, mempermudah iterasi dalam proses desain sistem kendali
termal satelit \cite{escobar2016}, serta mengoptimasi desain termal satelit
\cite{xiong2020}. Metode \textit{machine learning} digunakan untuk membuat
model regresi linear untuk memudahkan proses pengujian dan evaluasi model sehingga model
termal yang dihasilkan resilien terhadap perubahan dataset dan gangguan inheren
(\textit{noise}) dari dataset. 

Dengan demikian, diharapkan hasil akhir dari karya tulis ini adalah prosedur
pemodelan termal semi-empiris satelit LAPAN-A3 menggunakan model regresi linear
\textit{machine learning} yang dapat memprediksi suhu ke-7 \textit{node}
satelit (6 \textit{node} untuk setiap sisi satelit dan 1 \textit{node} untuk
plat tengah satelit) selama periode observasi 19 sampai dengan 20 Mei 2018.
Hasil prediksi suhu dari model termal satelit juga akan dievaluasi untuk
memvalidasi keakuratan model termal serta mengukur performa model saat ini.
Dengan demikian, pengembangan model termal satelit selanjutnya tidak lagi harus
memulai pemodelan dari awal, tapi dapat melanjutkan pengembangan prosedur dan
algoritma yang digunakan dalam karya tulis ini.

\section{Rumusan Masalah}

Dari latar belakang yang telah dijelaskan sebelumnya, dapat dibuat rumusan
masalah sebagai berikut :

\begin{enumerate}
\item Bagaimana pemodelan termal semi-empiris satelit LAPAN-A3 menggunakan metode \textit{machine learning} dapat dilakukan?
\item Bagaimana perbandingan perubahan suhu sisi-sisi satelit hasil prediksi model termal satelit dengan data telemetri satelit?
\item Bagaimana performa model termal satelit yang dihasilkan?
\end{enumerate}

\section{Tujuan Penelitian}

Berlandaskan rumusan masalah, penelitian pada karya tulis ini bertujuan untuk :

\begin{enumerate}
\item Melakukan penjabaran langkah-langkah untuk membuat model termal satelit \textit{node} banyak secara semi-empiris menggunakan metode \textit{machine learning}
\item Membuat model termal satelit yang dapat memprediksi perubahan suhu sisi-sisi satelit LAPAN-A3
\item Membandingkan perubahan suhu \textit{node} satelit hasil prediksi model termal dengan data telemetri satelit
\item Menganalisis performa hasil prediksi dari model termal satelit yang telah dibuat
\end{enumerate}

\section{Batasan Penelitian}

Penelitian dalam karya tulis ini dibatasi sebagai berikut :

\begin{enumerate}
\item Pemodelan termal satelit LAPAN-A3 dilakukan untuk periode observasi 19 dan 20 Mei 2018
\item Selama periode observasi, satelit dianggap tidak mengalami perubahan massa dan karakteristik termal
\item Satelit LAPAN-A3 dianggap berbentuk balok dan dibagi menjadi 7
	\textit{node} mewakili 6 sisi satelit dan plat tengah satelit dengan aturan
		konversi sumbu menjadi nomor \textit{node} sebagai berikut : X+ = 1, X- =
		2, Y+ = 3, Y- = 4, Z+ = 5, Z- = 6, serta plat tengah = 7
	\item Model dianggap akurat jika mayoritas skor koefisien determinasi ($R^2$) \textit{node} mencapai 0.95 dan mayoritas nilai \textit{root-mean-square error} (RMSE) \textit{node} di bawah 1 $^\circ$C
\end{enumerate}

Gambar \ref{fig:sumbua3} menunjukkan ilustrasi letak sumbu satelit LAPAN-A3.

\begin{figure}[H]
\setlength\belowcaptionskip{-0.7\baselineskip}
\begin{center}
\includegraphics[width=0.8\textwidth]{fig/sumbua3.png}
\caption{Ilustrasi letak sumbu satelit LAPAN-A3}
\label{fig:sumbua3}
\end{center}
\end{figure}


\section{Metodologi}

Gambar \ref{fig:metodologi} memuat metodologi pembuatan karya tulis ini. Secara
umum, penelitian dimulai dengan studi pustaka topik terkait pemodelan termal
satelit dan metode \textit{machine learning}. Selanjutnya, dilakukan
pengumpulan data yang dibutuhkan untuk pemodelan termal satelit. Kemudian,
dilakukan pembuatan dataset dalam format yang sesuai untuk digunakan dalam
pembuatan model \textit{machine learning}. Lalu, dataset tersebut dibagi
menjadi 2 set : set latihan untuk membantu model regresi linear \textit{machine
learning} dalam mengidentifikasi fitur dan tren perubahan suhu satelit dan set
ujian untuk mendapatkan prediksi perubahan suhu satelit dari model
\textit{machine learning}. Terakhir, hasil prediksi suhu satelit dari model
\textit{machine learning} dievaluasi untuk mengukur performa model termal saat ini.

\begin{figure}[H]
\setlength\belowcaptionskip{-0.7\baselineskip}
\begin{center}
\includegraphics[width=0.4\textwidth]{fig/graph_metodologi.png}
\caption{Metodologi penelitian karya tulis}
\label{fig:metodologi}
\end{center}
\end{figure}

Proses pembuatan dataset untuk model termal satelit memiliki algoritma
tersendiri yang disajikan pada Gambar \ref{fig:algoritma}. Secara singkat,
pembuatan dataset model termal satelit terdiri dari 3 langkah utama : penyiapan
dataset dasar, perhitungan faktor-faktor panas satelit, dan penyaringan
dataset. Penjelasan lebih menyeluruh terkait proses pembuatan dataset akan
dibahas pada bab Pembuatan Model Termal Satelit LAPAN-A3.

\begin{figure}[H]
\setlength\belowcaptionskip{-0.7\baselineskip}
\begin{center}
\includegraphics[width=0.75\textwidth]{fig/graph_algoritma.png}
\caption{Algoritma pembuatan dataset}
\label{fig:algoritma}
\end{center}
\end{figure}

\section{Sistematika Penulisan}

Sistematika penulisan karya tulis ini adalah sebagai berikut :

\begin{enumerate}
\item Bab 1 Pendahuluan

Bab ini membahas latar belakang, rumusan masalah, tujuan penelitian, batasan
penelitian, metodologi karya tulis, dan sistematika penulisan.

\item Bab 2 Tinjauan Pustaka

Bab ini membahas dasar teori yang digunakan dalam rangka pengerjaan karya tulis ini.

\item Bab 3 Pembuatan Model Termal Satelit LAPAN-A3

Bab ini menjabarkan langkah-langkah untuk membuat model termal semi-empiris
		satelit LAPAN-A3 dengan menggunakan metode \textit{machine learning}.

\item Bab 4 Hasil dan Analisis

Bab ini berisi hasil prediksi suhu sisi-sisi satelit LAPAN-A3 serta analisis performa model termal satelit LAPAN-A3 yang dihasilkan dari bab sebelumnya.

\item Bab 5 Kesimpulan dan Saran

Bab ini berisi kesimpulan dari penelitian yang telah dilakukan dan saran
untuk penelitian selanjutnya.
\end{enumerate}

%List of Symbols (Nomenclature)
% Latin first then Greek alphabet
% Capital (uppercase) first then lowercase

\nomenclature{\(C_{i}\)}{Kapasitas termal \textit{node i}}
\nomenclature{\(\Delta T_{i}\)}{Perubahan suhu \textit{node i}}
\nomenclature{\(\Delta t\)}{Selang waktu}
\nomenclature{\(\alpha_{s,i}\)}{Absorpsi surya total \textit{node i}}
\nomenclature{\(\eta\)}{Efisiensi listrik panel surya}
\nomenclature{\(F_{pg,i}\)}{Rasio luas panel surya terhadap luas \textit{node i}}
\nomenclature{\(E_{s}\)}{Penyinaran surya tegak lurus terhadap arah Matahari}
\nomenclature{\(A_{i}\)}{Luas \textit{node i}}
\nomenclature{\(\theta_{i}\)}{Sudut antara garis normal permukaan \textit{node i} dengan sinar Matahari}
\nomenclature{\(F_{e,i}\)}{Faktor gerhana \textit{node i}}
\nomenclature{\(\rho_{E}\)}{Albedo Bumi}
\nomenclature{\(F_{i,E}\)}{Nilai \textit{view factor} dari \textit{node i} ke Bumi}
\nomenclature{\(\beta\)}{Sudut antara bidang orbit satelit dengan vektor Matahari}
\nomenclature{\(F_{a}\)}{Faktor albedo satelit}
\nomenclature{\(F_{e}\)}{Faktor gerhana satelit}
\nomenclature{\(\varepsilon_{i}\)}{Emisivitas \textit{node i}}
\nomenclature{\(\varepsilon_{E}\)}{Emisivitas Bumi}
\nomenclature{\(T_{i}\)}{Suhu \textit{node i}}
\nomenclature{\(T_{j}\)}{Suhu \textit{node j}}
\nomenclature{\(\sigma\)}{Konstanta Stefan-Boltzmann}
\nomenclature{\(\sigma\)}{Standar deviasi}
\nomenclature{\(\mu\)}{Rata-rata data}
\nomenclature{\(x\)}{Nilai data}
\nomenclature{\(z\)}{Skor standar}
\nomenclature{\(F_{i,\infty}\)}{Nilai \textit{view factor} dari \textit{node i} ke lingkungan ruang angkasa}
\nomenclature{\(G_{ij}\)}{Koefisien kopling konduksi antara \textit{node i} dan \textit{node j}}
\nomenclature{\(R_{ij}\)}{Koefisien kopling radiasi antara \textit{node i} dan \textit{node j}}
\nomenclature{\(\dot{Q}_{dis,i}\)}{Laju masukan panas akibat disipasi elektrik \textit{node i}}
\nomenclature{\(\dot{Q}_{S,i}\)}{Laju masukan panas akibat radiasi dari Matahari ke \textit{node i}}
\nomenclature{\(\dot{Q}_{a,i}\)}{Laju masukan panas akibat albedo Bumi ke \textit{node i}}
\nomenclature{\(\dot{Q}_{E,i}\)}{Laju masukan panas akibat radiasi dari Bumi ke \textit{node i}}
\nomenclature{\(\dot{Q}_{\infty,i}\)}{Laju keluaran panas akibat disipasi panas dari \textit{node i} ke lingkungan ruang angkasa}
\nomenclature{\(\dot{Q}_{cond,ij}\)}{Laju masukan panas ke \textit{node i} akibat konduksi dari \textit{node j}}
\nomenclature{\(\dot{Q}_{rad,ij}\)}{Laju masukan panas ke \textit{node i} akibat radiasi dari \textit{node j}}
\nomenclature{\(I_{i}\)}{Arus sensor Matahari \textit{node i}}
\nomenclature{\(I_{0}\)}{Arus maksimum sensor Matahari satelit}
\nomenclature{\(c_{S}\)}{Koefisien suku panas akibat Matahari}
\nomenclature{\(c_{a}\)}{Koefisien suku panas akibat albedo}
\nomenclature{\(c_{E}\)}{Koefisien suku panas akibat Bumi}
\nomenclature{\(c_{env}\)}{Koefisien suku disipasi ke lingkungan ruang angkasa}
\nomenclature{\(F_{1,2}\)}{Nilai \textit{view factor} dari permukaan 1 ke permukaan 2}
\nomenclature{\(A_{1}\)}{Luas permukaan 1}
\nomenclature{\(A_{2}\)}{Luas permukaan 2}
\nomenclature{\(dA_{1}\)}{Elemen diferensial permukaan 1}
\nomenclature{\(dA_{2}\)}{Elemen diferensial permukaan 2}
\nomenclature{\(R_{12}\)}{Jarak antara elemen diferensial permukaan 1 dan 2}
\nomenclature{\(\Phi_1\)}{Sudut antara garis normal elemen diferensial permukaan 1 dengan garis jarak antara elemen diferensial permukaan 1 dan 2}
\nomenclature{\(\Phi_2\)}{Sudut antara garis normal elemen diferensial permukaan 2 dengan garis jarak antara elemen diferensial permukaan 1 dan 2}
\nomenclature{\(\hat{n}\)}{Vektor normal permukaan plat}
\nomenclature{\(r\)}{Jari-jari bola}
\nomenclature{\(H\)}{Jarak antara plat dan bola}
\nomenclature{\(\gamma\)}{Sudut antara vektor normal permukaan plat terhadap garis jarak antara plat dan bola}
\nomenclature{\(\phi\)}{Posisi sudut satelit terhadap titik \textit{sub-solar}}
\nomenclature{\(\phi_{es}\)}{Posisi sudut satelit terhadap titik \textit{sub-solar} saat memasuki fase gerhana}
\nomenclature{\(R^2\)}{Koefisien determinasi}
\nomenclature{\(RMSE\)}{Root Mean Squared Error}
\nomenclature{\(Y\)}{Matriks variabel dependen}
\nomenclature{\(X\)}{Matriks variabel independen}
\nomenclature{\(w\)}{Matriks koefisien yang tidak diketahui}
\nomenclature{\(\varepsilon\)}{Matriks kesalahan atau gangguan}

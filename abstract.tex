\begin{center}
       \Large
       \textbf{ABSTRACT} \\
        \vspace{1.5cm}
        \large
        \textbf{Semi-empirical Thermal Modelling of LAPAN-A3 Satellite Using Machine Learning Method}\\
        
        \vspace{1.5cm}
        
        by\\
        Ricky Sutardi\\
        NIM : 13617051\\
        (Bachelor's Program in Aerospace Engineering)\\
        \vspace{1.5cm}
\end{center}

From the satellite temperature sensor readings, it is found that LAPAN-A3, an
Indonesian microsatellite, requires special maneuver to keep its main payload,
a multispectral imager, above 0 \degree C. It can then be concluded that the
satellite passive thermal control system is inadequate for the satellite
mission needs. To prevent this problem in future satellite thermal design, a
new simple thermal model which can model LAPAN-A3 thermal characteristic
accurately is needed.

Conventional thermal analysis requires accurate calculations of many variable
which results in increased complexity. Meanwhile, data-driven approaches in
solving satellite thermal problem are becoming increasinly common to reduce
the complexity of thermal modelling. Thus, a simple thermal model which can
deduce the needed satellite thermal parameters to predict the satellite
temperature from existing operational satellite data hopefully can result in a
quicker and easier thermal modelling process.

In this paper, a semi-empirical 7-node thermal model of LAPAN-A3 satellite is
developed using machine learning method. The model is trained with satellite
data for the observation period of 19 and 20 May 2018. Linear regression using
machine learning is chosen to reduce the number of variables which must be
calculated to model the satellite. Initial evaluation shows that the thermal
model is generally able to predict LAPAN-A3 node temperature changes and has
potential usage in real-life satellite development. Even though the model can
still be improved upon, it may serve as a basis for future satellite thermal
model.

\vspace{1.0cm}
\noindent 
Keywords : thermal model, satellite, machine learning, linear regression, LAPAN-A3 
